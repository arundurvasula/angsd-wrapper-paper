\documentclass[12pt]{article}
\usepackage{geometry} 
\usepackage{indentfirst}
\usepackage{hyperref}
\usepackage{color}
\usepackage{comment}
\usepackage[pdftex]{graphicx}  
\usepackage{caption}
\usepackage{natbib}
\usepackage{mathtools}
\usepackage{units}
\usepackage{booktabs}
\usepackage{authblk}
\renewcommand{\baselinestretch}{1.5}
\geometry{a4paper} 
\bibliographystyle{apalike}

%----------------------------------------
%AUTHORS
%----------------------------------------
\title{angsd-wrapper: a utility to aid in the analysis of next generation sequencing data}
\author[1]{Arun Durvasula}
\author[2]{Peter L. Morrell}
\author[1,3]{Jeffrey Ross-Ibarra}

\affil[1]{Department of Plant Sciences, University of California Davis}
\affil[2]{Department of Agronomy and Plant Genetics, University of Minnesota}
\affil[3]{Center for Population Biology and Genome Center, University of California Davis}

\date{}

%-----------------------------------------------------------------------------------------------------------------
% BEGIN DOCUMENT
%-----------------------------------------------------------------------------------------------------------------
\begin{document}
\maketitle

\begin{abstract}
blah

\textbf{Availability:} github
\end{abstract}

\section*{Introduction}
The advent of highly multiplexed sequencing has brought about a large amount of new data more quickly than has been anticipated. 
This change allows biologists to use many more samples and brings with it the power to make more confident and better statistical inferences.  %need to cite some examples of papers doing this. with lots of samples to low N. can cite miaze hapmap (4x) and rice HUang 2013 (2x) and probably a lot of other papers where they do large n and small depth. should cite some from both plants and animals.
However this also presents statistical challenges and new methods must be created with this amount of data in mind. 
In order to aid in the analysis of next generation sequencing data, population genetics software such as ANGSD has been developed. 
This program allows the user to calculate population genetic statistics such as site frequency spectrums, neutrality tests, and theta estimators from sequence data aligned to a reference. 
ANGSD uses likelihood based approaches to make full use of the data afforded by highly multiplexed sequencing. 
This has been used several studies to calculate summary statistics ( CITATIONS ). 
However, ANGSD requires much familiarity with command line tools and remains inaccessible to biologists that are not from a computational background. 

Here we present a software package that aids in the preparation of analysis and facilitates analysis with interactive graphing software implemented in R and Shiny. 
Before analysis, angsd-wrapper turns a multistep analysis such as calculating Tajima's D into one step where the information needed is supplied using a configuration file. 
After ANGSD has finished the calculations, a browser based application can be used to interactively graph the results.

\section*{Implementation}

\subsection*{Pre-analysis}
angsd-wrapper is implemented using several bash scripts that call ANGSD methods and handle saving intermediate files between the initial data preparation and the final data analysis. 
Each overall method in ANGSD, such as calculating estimates of $\theta$, follows a specific order of program calls. 
Thus, we have abstracted away the running of each step and provided a set of default values for parameters and instead require the user to supply the data using a configuration file. 
The user can override the default values of the parameters in the configuration file as well. 

\subsection*{Post-analysis}
angsd-wrapper contains a powerful graphing application based on R and Shiny. %sure you don't want to call it "wangsd" ? 
After the analysis is done by ANGSD,  the user can load the resulting statistics into a web-based application hosted on the user's computer. 
This application allows the interactive plotting of values such as estimators of $\theta$ and Tajima's D as well as the ability to load gene annotations from Ensembl. 
These features make it easy and intuitive to analyze next generation sequencing data.
% need to explain how shiny is used, that it calls ggplot and shiny, checks and installs, etc.


\subsection*{Example analyses}
%present example analyses. a figure should show config file i think and another a screenshot of shiny plots

\end{document}
